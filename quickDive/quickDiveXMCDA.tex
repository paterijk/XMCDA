\documentclass[a4paper,oneside,10 pt]{article}

\usepackage{longtable,geometry}
% \usepackage[frenchb]{babel}
\usepackage[utf8]{inputenc}
\usepackage[T1]{fontenc}
\usepackage{ascii}
\usepackage{verbatim}
\usepackage{listings}
\usepackage[usenames]{color}
\usepackage{xspace}
\usepackage[pdftex,
	draft = false,				% Mettre true pour ne pas générer les liens! 
	bookmarks = true,			% Signets
	bookmarksnumbered = true,		% Signets numérotés
% 	pdfpagemode = None,			% Signets/vignettes fermé à l'ouverture
	pdfstartview = FitV,			% La page prend toute la hauteur (remplacer V par H pour la largeur)
	pdfpagelayout = SinglePage,		% Vue par page
	colorlinks = true,			% Liens en couleur
	urlcolor = blue,			% Couleur des liens externes
	linkcolor = blue,			% Couleur des liens normaux
	citecolor = blue,			% couleur des citations
	pdfborder = {0 0 0},			% Style de bordure : ici, pas de bordure
 	plainpages = false,			% necessaire si on veut des liens retour
						% corrects vers les citations
	pagebackref				% des liens retour vers 
						%l'endroit ou la citation a eu lieu
	]{hyperref}
\usepackage{url}
\usepackage{multicol}

\definecolor{bproto}{rgb}{0.92,0.92,0.92}
\definecolor{bexample}{rgb}{1,1,1}

\lstset{numbers=left,numberstyle=\tiny,stepnumber=3,firstnumber=1,language=XML,basicstyle=\small,columns=flexible,showstringspaces=false,emph={uint8_t,uint16_t,uint32_t,size_t,sig_atomic_t,time_t,clock_t,sigset_t,restrict},emphstyle=\bfseries,escapeinside={/*@}{@*/}}

\lstdefinestyle{prototype}{numbers=none,backgroundcolor=\color{bproto},xleftmargin=2em}

\lstdefinestyle{example}{numbers=none,xleftmargin=2em}

\geometry{dvips,a4paper,margin=1in}

\newcommand{\XMCDA}{{\asciifamily XMCDA}\xspace}
\newcommand{\XML}{{\asciifamily XML}\xspace}
\newcommand{\MCDA}{MCDA\xspace}
\newcommand{\code}{\asciifamily}

\title{\begin{huge}Quick dive into \XMCDA-2.0\end{huge}}

\author{Raymond Bisdorff\thanks{Université du Luxembourg, FSTC - ILIAS - CSC, Decision Systems Group, 6, rue Richard Coudenhove-Kalergi, L-1359 Luxembourg} \and Patrick Meyer\thanks{Institut Télécom, Télécom Bretagne, UMR CNRS 3192 Lab-STICC, Technopôle Brest-Iroise CS 83818, F-29238 Brest Cedex 3, Université européenne de Bretagne, {\asciifamily patrick.meyer@telecom-bretagne.eu}}
\and Thomas Veneziano\footnotemark[1]}

\date{For a complete documentation about \XMCDA, please visit  \url{http://www.decision-deck.org/xmcda}}

\begin{document}

\maketitle

\begin{abstract}
This article presents the main features of \XMCDA, a standardised \XML proposal to represent {\em objects} and {\em data} issued from the field of Multiple Criteria Decision Aid (\MCDA). Its main objective is to allow different \MCDA algorithms to interact and to be easily callable from a software like, e.g., the diviz platform of the Decision Deck project\footnote{\url{http://www.decision-deck.org}}. We present the structure of \XMCDA and detail the main underlying data structures via examples speaking for themselves. Note that this document does not replace a detailed documentation of the related \XML schema.
\end{abstract}

\tableofcontents


%%%%%%%%%%%%%%%%%%%%%%%%%%%%%%%%%%%%%%%%%%%%%%%%%%%%%%%%%%%%%%%%%%%%%%%%%%%%
%%%%%%%%%%%%%%%%%%%%%%%%%%%%%%%%%%%%%%%%%%%%%%%%%%%%%%%%%%%%%%%%%%%%%%%%%%%%


\section{Introductory considerations}

The abstract description of the \XMCDA structure is performend via a detailed \XML schema. This document does not intend to replace the documentation of this schema, but it is rather meant to present the \XMCDA structure in a practical and natural way. In order to motivate as many programers as possible to adopt the \XMCDA representation, we have decided to express as many \MCDA concepts as possible through a few general data structures coded in \XML. In this section we first present the conventions used on the tag names, and then present three \XML attributes which are used in all the data structures of \XMCDA. 

To avoid misunderstandings, note the following conventions which are used in this document: 

\begin{itemize}
 \item The term \MCDA \textit{concept} describes an real or abstract construction related to the field of \MCDA which needs to be stored in \XMCDA (like, for example, the importance of the criteria);
 \item The term \XMCDA \textit{type} stands for an \XML structure that we created for the purpose of \XMCDA (like, for example, {\code criteriaValues} to store general values related to a set of criteria). 
\end{itemize}

\subsection{On the names of the tags}

By convention, the name of a tag starts by a lower-case letter. The rest of the name is in mixed case with the first letter of each internal word capitalised. This allows to easily read and understand the meaning of a tag. We use whole words and avoid as much as possible acronyms and abbreviations. Consider for example the tagnames {\code methodOptions}, {\code performanceTable} and    {\code criterionValue}. 
 Note that objects of the same \XMCDA type can in general be gathered in a compound tag, represented by a single {\code XML} tag named after the plural form of its components (e.g., {\code alternatives}). 

\subsection{On the attributes of the tags}

The three following attributes can be found in any of the main data tags: {\code id},  {\code name} and {\code mcdaConcept}. They are in general optional, except for the {\code id} attribute in the description of an alternative, a criterion or a category (see Section \ref{sec:basicTypes} for further details). Each of these three attributes has a particular purpose in  \XMCDA:

\begin{itemize}
 \item The {\code id} attribute allows to identify an object with a {\em machine readable} code or identifier. As an illustration consider the following set of two alternatives {\code a03} and {\code a04} which is identified by {\code set1}. 
{\code
\begin{lstlisting}[style=prototype]
 <alternativesSet id="set1">
	 <element>
		 <alternativeID>a03</alternativeID>
	 </element>
	 <element>
		 <alternativeID>a04</alternativeID>
	 </element>
 </alternativesSet>
\end{lstlisting}
}
 \item The {\code name} attribute allows to give a \textit{human-readable} name to a particular object. As an illustration consider the following code, which shows how a named parameter could be passed to an \MCDA method.
{\code
\begin{lstlisting}[style=prototype]
<parameter id="numIt" name="number of iterations">
	 <value><integer>3</integer></value>
 </parameter>
\end{lstlisting}
}
 \item The {\code mcdaConcept} attribute allows to identify the \MCDA concept linked to a particular instance of an \XMCDA type. To illustrate this, consider the following example, where the set of alternatives $\{\mbox{a03}, \mbox{a04}\}$ is considered as a kernel of a graph.
{\code
\begin{lstlisting}[style=prototype]
 <alternativesSet mcdaConcept="kernel" name="a kernel with two elements">
	 <element>
		 <alternativeID>a03</alternativeID>
	 </element>
	 <element>
		 <alternativeID>a04</alternativeID>
	 </element>
 </alternativesSet>
\end{lstlisting}
}
\end{itemize}

After these preliminary considerations, we present in the following section the general structure of an \XMCDA file, before turning to a detailed description of the fundamental \XMCDA types in Section \ref{sec:basicTypes}. 

\section{Outline of the structure of an \XMCDA file}\label{sec:generalStructure}

An \XMCDA file may contain several tags under the root element. These tags allow to describe various \MCDA related data from the a few general categories:

\begin{itemize}
 \item Project or file description;
 \item Output messages from methods (log or error messages) and input information for methods (parameters);
 \item Description of major \MCDA concepts as attributes, criteria, alternatives, categories;
 \item The performance table;
 \item Further preferential information related to criteria, alternatives, attributes or categories.
\end{itemize}

We detail these general categories later in Section~\ref{sec:additionalTypes}. However, first we present in the following section, the fundamental data structures which are defined in \XMCDA. They represent the basis of any \XML file which respects the \XMCDA standard. 


%%%%%%%%%%%%%%%%%%%%%%%%%%%%%%%%%%%%%%%%%%%%%%%%%%%%%%%%%%%%%%%%%%%%%%%%%%%%
%%%%%%%%%%%%%%%%%%%%%%%%%%%%%%%%%%%%%%%%%%%%%%%%%%%%%%%%%%%%%%%%%%%%%%%%%%%%


\section{Elementary \XMCDA types}\label{sec:basicTypes}

% \subsection{Elementary types}

\subsection{Values}

To store a \textit{value}, we use the \XMCDA type {\code value} which can represent an integer, a real number, an interval, a rational, a nominal value, an ordinal value, a \textit{not available} value or a {\em binary64} string. The following listing shows a set of values containing 4 elements of different types. 
{\code
\begin{lstlisting}[style=prototype]
 <values>
	 <value><integer>8</integer></value>

	 <value><rankedLabel>
	 	 <label>Good</label>
		 <rank>1</rank>
	 </rankedLabel></value>

	 <value><rational>
		 <numerator>10</numerator>
		 <denominator>3</denominator>
	 </rational></value>

 	<value><real>3.141526</real></value>
 </values> 
\end{lstlisting}
}
Note that there also exists an \XMCDA type called {\code numericValue} which restricts {\code value} to numeric values. 

\subsection{Intervals}

{\code lowerBound} and {\code upperBound} tags are of the type {\code value}, so they can represent an interval of numeric or ordinal values which are coded as follows.
{\code
\begin{lstlisting}[style=prototype]
 <interval>
	 <lowerBound><integer>0</integer></lowerBound>
	 <upperBound><integer>5</integer></upperBound>
 </interval>
\end{lstlisting}

\begin{lstlisting}[style=prototype]
 <interval>
	 <lowerBound><rankedLabel>[..]</rankedLabel></lowerBound>
	 <upperBound><rankedLabel>[..]</rankedLabel></upperBound>
 </interval>
\end{lstlisting} 
}
\subsection{Points}

Some more complex \XMCDA types, as, e.g., {\code function}, require the concept of \textit{point} which is represented by the following code. Note that the abscissa as well as the ordinate are of the type {\code value}. 
{\code
\begin{lstlisting}[style=prototype]
 <point>
	 <abscissa><real>2.7182818</real></abscissa>
	 <ordinate><integer>23</integer></ordinate>
 </point>
\end{lstlisting}
}

\subsection{Scales}

The concept of {\em scale} is described as follows. Note that a scale may be {\em quantitative}, {\em qualitative} or {\em nominal} scale.
{\code
\begin{lstlisting}[style=prototype]
 <scale>
	 <quantitative>
		 <minimum><real>0.00</real></minimum>
		 <maximum><real>1.00</real></maximum>
	 </quantitative>
 </scale>
\end{lstlisting}
}

\subsection{Functions}

A {\code function} can either be a constant, a linear, a piecewise linear function or simply a set of points. Note that the functions appear in more complex types related to the criteria. 

{\code
\begin{lstlisting}[style=prototype]
 <function>
	 <constant><real>456.3847</real></constant>
 </function>

 <function>
	 <linear>
		 <slope><real>4.00</real></slope>
		 <intercept><real>4.00</real></intercept>
	 </linear>
 </function>

 <function>
	 <points>[..]</points>
 </function>
\end{lstlisting}
}

\subsection{Description}

Each tag defined in {\code XMCDA} owns a description which allows to clearly describe it. A short example is given hereafter for a list of alternatives. 
{\code
\begin{lstlisting}[style=prototype]
 <alternatives>
	 <description>
		 <title>The list of alternatives</title>
		 <comment>Only european cars are considered.</comment>
	 </description>
	[..]
 <alternatives>
\end{lstlisting}
}
The description tag is used to add some comments on a value, a criterion, etc., or to specify, e.g., the author of a piece of information. All the tags are optional. 


%%%%%%%%%%%%%%%%%%%%%%%%%%%%%%%%%%%%%%%%%%%%%%%%%%%%%%%%%%%%%%%%%%%%%%%%%%%%
%%%%%%%%%%%%%%%%%%%%%%%%%%%%%%%%%%%%%%%%%%%%%%%%%%%%%%%%%%%%%%%%%%%%%%%%%%%%


\section{Detailed structure of an \XMCDA file}\label{sec:additionalTypes}

In Section~\ref{sec:basicTypes} we have presented the elementary \XMCDA types which are very general and can therefore be used in various situations. 
In this section we give a more detailed description of the different information categories presented in Section~\ref{sec:generalStructure} and which reuse the fundamental \XMCDA types.

\subsection{Method and project specific data}

\subsubsection{How to describe the current project?}

The tag {\code projectReference} can be used to describe the current project by different tags from the {\code description} type which was explained earlier. The following code gives a short example of such a description. 
{\code
\begin{lstlisting}[style=prototype]
 <projectReference id="testProblem">
	<version>1.2</version>
	<creationDate>2008-10-20T22:24:02</creationDate>
	<author>Patrick Meyer and Thomas Veneziano</author>
 </projectReference>
\end{lstlisting}
}

\subsubsection{How to implement method-specific parameters?}

Some methods require some specific parameters in order to guide the resolution of a decision problem. Those parameters can be specified by the {\code methodParameters} tag as follows. Notice that a parameter can be either a value or a function.

{\code
\begin{lstlisting}[style=prototype]
 <methodParameters>
	 <approach>outranking</approach>
	 <problematique>choice</problematique>
	 <methodology>Rubis</methodology>
	 <parameter name="variant">
	 	<value><label>standard</label></value>
	 </parameter>
 </methodParameters>
\end{lstlisting}
}

\subsubsection{How to store method-specific messages?}

Certain methods might generate some error or log messages. These can be stored in the {\code methodMessages} tag. 

{\code
\begin{lstlisting}[style=prototype]
 <methodMessages>
	 <errorMessage name="Error 404">
		 <number>404</number>
		 <message>Data not found. Did you specify a bad file name?</message>
	 </errorMessage>
	 <logMessage>
		 <number>0</number>
		 <message>Execution successful.</message>
	 </logMessage>
 </methodMessages>
\end{lstlisting}
}

\subsection{Definition of alternatives, criteria, attributes and categories}

\subsubsection{How to define alternatives?}

Alternatives are defined and described under the {\code alternatives} tag. They can be either {\code active} or not and either be {\code real} or {\code fictive} alternatives. In addition, they can also be flagged as {\code reference} alternatives (for profiles in a sorting problem, e.g.). The {\code id} of an alternative is mandatory.  

{\code
\begin{lstlisting}[style=prototype]
 <alternatives name="myAlternatives">
	 <alternative id="x1" name="Red Ferrari"/>
	 <alternative id="x2" name="Blue Corvette">
		 <type>real</type>
		 <active>true</active>
		 <reference>false</reference>
	 </alternative>
	 <alternative id="x3" name="UFO">
		 <type>fictive</type>
	 </alternative>
  </alternatives>
\end{lstlisting}
}

Note that it is possible to define sets of alternatives under the {\code alternativesSets} tag (see Section~\ref{subsec:xSet} for further details).

\subsubsection{How to define criteria?}

Criteria are defined and described under the {\code criteria} tag. For each criterion one has to define its {\code id}. 

In the following example, the first criterion $g1$ represents the power of a car. 

{\code
\begin{lstlisting}[style=prototype]
 <criteria>
	  <criterion id="g1">
		 <description>
			 <comment>Power in horsepowers</comment>
		 </description>
		 <attributeReference>att1</attributeReference>
		 <scale>
			 <quantitative>
				 <preferenceDirection>max</preferenceDirection>
				 <minimum><real>50</real></minimum>
				 <maximum><real>200</real></maximum>
			 </quantitative>
		 </scale>
	 </criterion>
	 <criterion id="g2"/>
  </criteria>
\end{lstlisting}
}

Note that criteria can be linked to other criteria (or attributes) via a {\code criteriaReference} (or an {\code attributeReference} tag. Note that it is possible to define sets of criteria under the {\code criteriaSets} tag (see Section~\ref{subsec:xSet} for further details).

\subsubsection{How to define attributes?}

Attributes are defined the same way as criteria under the {\code attributes} tag and can also be linked to other attributes (or criteria). 

\subsubsection{How to define categories?}

Categories are defined under the tag {\code categories} as shown in the following example. 
{\code
\begin{lstlisting}[style=prototype]
 <categories>
	 <category id="g" name="goodStudents">
		 <active>true</active>
	 </category>
	 <category id="m" name="mediumStudents">
		 <active>false</active>
	 </category>
  </categories>
\end{lstlisting}
}

Note that it is possible to define sets of categories under the {\code categoriesSets} tag (see Section~\ref{subsec:xSet} for further details).


%%%%%%%%%%%%%%%%%%%%%%%%%%%%%%%%%%%%%%%%%%%%%%%%%%%%%%%%%%%%%%%%%%%%%%%%%%%%


\subsection{The performance table}

The performance table is defined and described under the tag {\code performanceTable}. It contains, for each alternative (given by its id), a list of performances, given by a criterion id (or attribute id) and a corresponding performance value. 

{\code
\begin{lstlisting}[style=prototype]
 <performanceTable>
	 <alternativePerformances>
		 <alternativeID>alt1</alternativeID>
		 <performance>
			 <criterionID>g1</criterionID>
			 <value><real>72.10</real></value>
			 </performance>
		 <performance>
			 <criterionID>g2</criterionID>
			 <value><real>82.62</real></value>
		 </performance>
	 </alternativePerformances>
	 <alternativePerformances>
		 <alternativeID>alt2</alternativeID>
		 [..]
	 </alternativePerformances>
 </performanceTable>
\end{lstlisting}
}

%%%%%%%%%%%%%%%%%%%%%%%%%%%%%%%%%%%%%%%%%%%%%%%%%%%%%%%%%%%%%%%%%%%%%%%%%%%%

\subsection{Advanced information on alternatives, criteria, attributes and categories}\label{sec:advancedTypes}

Let us now present some more advanced \XMCDA tags which allow to represent many structures issued from the field of \MCDA. 

To simplify the presentation of \XMCDA, we have defined a few generic structures which we have adapted for alternatives, criteria, attributes and categories. To avoid some redundant explanations and notation, we write {\code x}Set for the generic structure related to the \XMCDA types {\code alternativesSet}, {\code criteriaSet}, {\code attributesSet} and {\code categoriesSet}. The same convention is used for the {\code x}Value, {\code x}LinearConstraint, {\code x}Comparisons and {\code x}Matrix types, described in the following subsections. Note that all those tags are defined directly under the root \XMCDA tag.

\subsubsection{{\code x}Set}\label{subsec:xSet}

An {\code x}Set is a set of elements. Each of the elements, as well as the whole set, can be valued. The following code represents a set of alternatives, where one alternative is valued (e.g., by the credibility of its membership to the set), and where the whole set is valued by two {\em qualities}.

{\code
\begin{lstlisting}[style=prototype]
<alternativesSet id="good1" mcdaConcept="goodChoice">
	 <element>
		 <alternativeID>a03</alternativeID>
	 </element>
	 <element>
		 <alternativeID>a04</alternativeID>
		 <value><real>0.88</real></value>
	 </element>
	 <values name="qualities">
		 <value name="independence"><real>100.00</real></value>
		 <value name="outranking"><real>74.00</real></value>
	 </values>
 </alternativesSet>
\end{lstlisting}
} 

\subsubsection{{\code x}Value}

An {\code x}Value is a value associated with an object or a set of objects. 
{\code
\begin{lstlisting}[style=prototype]
 <alternativeValue mcdaConcept="overallValue">
	 <alternativeID>alt1</alternativeID>
	 <value>[..]</value>
 </alternativeValue>

 <criterionValue>
	 <criteriaSetID>cs3</criteriaSetID>
	 <value>[..]</value>
 </criterionValue>
\end{lstlisting}
}

Note that if the set has not been defined earlier, it is possible to define it here. 
{\code
\begin{lstlisting}[style=prototype]
 <categoryValue mcdaConcept="cardinality">
	<categoriesSet>
		<element><categoryID>cat1</categoryID></element>
		<element><categoryID>cat2</categoryID></element>
	</categoriesSet>
	 <value>[..]</value>
 </categoryValue>
\end{lstlisting}
}

\subsubsection{{\code x}LinearConstraints}

The following example gives us the representation of the constraint
\begin{eqnarray*}
	2\cdot \mbox{weight}(c_{2})+\mbox{weight}(c_{4}) \leq 0.5
\end{eqnarray*}
{\code
\begin{lstlisting}[style=prototype]
 <criteriaLinearConstraints mcdaConcept="weight">
	 <constraint name="a strange constraint">
		 <constraintNumber>4</constraintNumber>
		 <element>
			 <criterionID>c2</criterionID>
			 <coefficient><real>2.00</real></coefficient>
		 </element>
		 <element>
			 <criterionID>c4</criterionID>
			 <coefficient><real>1.00</real></coefficient>
		 </element>
		<operator>leq</operator>
		<rhs>0.5</rhs>
	 </constraint>
 <criteriaLinearConstraints>
\end{lstlisting}
}

The {\code operator} tag can either be {\code eq} ($=$), {\code leq} ($\leq$) or {\code geq} ($\geq$). 

\subsubsection{{\code x}Comparisons}

An {\code x}Comparisons allows to represent valued binary relations on criteria, alternatives, categories and attributes. A tag {\code valuation} of type {\code xmcda:scale} can be used to determine the scale of the valuation and the tag {\code relationType} allows to express what kind of relation is stored (we recommend to use keywords like \textit{preference}, \textit{indifference}, \textit{incomparability}, \textit{outranking}, \textit{geq}, \textit{leq}, \textit{eq}, \textit{neq}, \textit{gtr}, \textit{less}, or any personnalised strings). 

{\code
\begin{lstlisting}[style=prototype]
 <alternativesComparisons mcdaConcept="outrankingDigraph" name="Stilde">
	<valuation>[..]</valuation>
	<relationType>preference</relationType>
 	<pairs>
		<pair>
			<initial><alternativeID>a01</alternativeID></initial>
			<terminal><alternativeID>a01</alternativeID></terminal>
			<value><real>0.00</real></value>
		</pair>
		<pair>
			<initial><alternativeID>a01</alternativeID></initial>
			<terminal><alternativeID>a02</alternativeID></terminal>
			<value><real>1.00</real></value>
		</pair>
 	</pairs>
 </alternativesComparisons>
\end{lstlisting}
}

\subsubsection{{\code x}Matrix}

An {\code x}Matrix allows to represent matrixes on criteria, alternatives, attributes and categories. A {\code scale} can be defined to determine the domain of the valuation. 

{\code
\begin{lstlisting}[style=prototype]
 <criteriaMatrix mcdaConcept="correlationTable">
     <row>
         <criterionID>g01</criterionID>
         <column>
             <criterionID>g01</criterionID>
             <value>
                 <real>1.00</real>
             </value>
         </column>
         <column>
             <criterionID>g02</criterionID>
             <value>
                  <real>-0.33</real>
             </value>
         </column>
     </row>
     <row>
         <criterionID>g02</criterionID>
         <column>
             <criterionID>g01</criterionID>
             <value>
                <real>-0.33</real>
             </value>
         </column>
         <column>
             <criterionID>g02</criterionID>
             <value>
                <real>1</real>
             </value>
         </column>
     </row>
 </criteriaMatrix>
\end{lstlisting}
}

For preferential information related to categories, we have defined the three supplementary tags {\code categoryProfile}, {\code categoriesContents} and {\code alternativesAffectations}.

\subsubsection{Categories profiles}

The tag {\code categoryProfile} is used to describe the caracteristics of a category via \textit{central} or \textit{limit} profiles, as shown in the following piece of code. 
{\code
\begin{lstlisting}[style=prototype]
 <categoriesProfiles>
	 <categoryProfile>
		 <alternativeID>alt3</alternativeID>
		 <central>
			 <categoryID>cat4</categoryID>
			 <value><real>0.354</real></value>
		 </central>
	 </categoryProfile>
	 <categoryProfile>
		 <alternativeID>alt1</altenativeID>
		 <limits>
			 <lowerCategory>
				 <categoryID>medium</categoryID>
				 <value><real>1.00</real></value>
			 </lowerCategory>
			 <upperCategory>
				 <categoryID>good</categoryID>
				 <value><real>0.678</real></value>
			 </upperCategory>
		 </limits>
	 </categoryProfile>
 </categoriesProfiles>
\end{lstlisting}
}

\subsubsection{Categories contents}

The tag {\code categoriesContents} allows to store the content of each category in terms of alternatives belonging to it. 
{\code
\begin{lstlisting}[style=prototype]
 <categoriesContents>
	 <categoryContent>
		<categoryID>cat1</categoryID>
		<alternativesSet>
			<element>
				<alternativeID>alt3</alternativeID>
				<value><real>0.89</real></value>
			</element>
			<element>
				<alternativeID>alt4</alternativeID>
			</element>
		</alternativesSet>
	 </categoryContent>
 </categorieContents>
\end{lstlisting}
}

\subsubsection{Alternatives affectations}

Finally, the tag {\code alternativesAffectations} allows to store which alternative belongs to which category (or set of categories). 
{\code
\begin{lstlisting}[style=prototype]
 <alternativesAffectations>
	<alternativeAffectation>
		<alternativeID>alt2</alternativeID>
		<categoryID>cat03</categoryID>
	</alternativeAffectation>
	<alternativeAffectation>
		<alternativeSetID>alts3</alternativeSetID>
		<categoriesSetID>catSet13</categoriesSetID>
	</alternativeAffectation>
	<alternativeAffectation>
		<alternativeID>alt4</alternativeID>
		<categoriesInterval>
			<lowerBound><categoryID>medium</categoryID></lowerBound>
			<upperBound><categoryID>veryGood</categoryID></upperBound>
		</categoriesInterval>
	 </alternativeAffectation>
 </alternativesAffectation>
\end{lstlisting}
}

Finally, to specify a hierarchy of concepts (criteria, alternatives, attributes and categories), we have defined the {\code hierarchy} tag.

%%%%%%%%%%%%%%%%%%%%%%%%%%%%%%%%%%%%%%%%%%%%%%%%%%%%%%%%%%%%%%%%%%%%%%%%%%%%

\subsubsection{Specifying a hierarchy of concepts}

The following code shows a hierarchy of criteria. Each node can contain one or more values. 

{\code
\begin{lstlisting}[style=prototype]
 <hierarchy>
    <description>
      <comment>A hierarchy of criteria</comment>
    </description>
    <node>
      <criterionID>economical</criterionID>
      <node>
        <criterionID>maintenance</criterionID>
      </node>
      <node>
        <criterionID>price</criterionID>
      </node>
    </node>
    <node>
      <criterionID>ecological</criterionID>
      <node>
        <criterionID>CO2</criterionID>
      </node>
      <node>
        <criterionID>Cx</criterionID>
      </node>
    </node>
  </hierarchy>
\end{lstlisting}
}

%%%%%%%%%%%%%%%%%%%%%%%%%%%%%%%%%%%%%%%%%%%%%%%%%%%%%%%%%%%%%%%%%%%%%%%%%%%%
%%%%%%%%%%%%%%%%%%%%%%%%%%%%%%%%%%%%%%%%%%%%%%%%%%%%%%%%%%%%%%%%%%%%%%%%%%%%


\section{Examples}

\subsection{The output of the Rubis web service}
\begin{multicols}{2}
{\code
\begin{lstlisting}[style=prototype, basicstyle=\tiny]
<?xml version="1.0" encoding="UTF-8"?>

<?xml-stylesheet href="xmcdaXSL.xsl" type="text/xsl" ?>
<?xml-stylesheet href="cssStyle.css" type="text/css" ?>

<xmcda:XMCDA xmlns:xmcda="http://www.decision-deck.org/2009/XMCDA-2.0.0"
      xmlns:xsi="http://www.w3.org/2001/XMLSchema-instance"
      xsi:schemaLocation="http://www.decision-deck.org/2009/XMCDA-2.0.0
      file:../XMCDA-2.0.0.xsd">

  <projectReference>
    <title>Rubis Best Choice Recommendation</title>
    <subTitle>Performance Tableau in XMCDA format.</subTitle>
    <author>digraphs Module (RB)</author>
    <version>saved from Python session</version>
  </projectReference>
    
  <methodParameters id="Rubis" name="Rubis best choice method" 
  	mcdaConcept="methodData">
    <description>
      <subTitle>Method data</subTitle>
      <comment>Rubis best choice recommendation in XMCDA format.</comment>
      <version>1.0</version>
    </description>
    <parameter name="variant">
      <value><label>standard</label></value>
    </parameter>
    <parameter name="valuationType">
      <value><label>bipolar</label></value>
    </parameter>
  </methodParameters>
    
  <alternatives mcdaConcept="actions">
    <description>
      <title>List of Alternatives</title>
      <subTitle>Potential decision actions.</subTitle>
    </description> 
    <alternative id="a01" name="random expensive decision action">
      <type>real</type>
      <active>true</active>
    </alternative>
    <alternative id="a02" name="random expensive decision action">
      <type>real</type>
      <active>true</active>
    </alternative>
    <alternative id="a03" name="random neutral decision action">
      <type>real</type>
      <active>true</active>
    </alternative>
  </alternatives>
    
  <criteria>
    <description>
      <title>Rubis family of criteria.</title>
    </description>
    
    <criterion id="g01" name="random benefit criterion">
      <active>true</active>
      <scale>
        <qualitative>
          <rankedLabel>
            <rank>1</rank>
            <label>good</label>
          </rankedLabel>
          <rankedLabel>
            <rank>2</rank>
            <label>medium</label>
          </rankedLabel>
          <rankedLabel>
            <rank>3</rank>
            <label>bad</label>
          </rankedLabel>
        </qualitative>
      </scale>
      <thresholds>
        <threshold mcdaConcept="ind">
          <constant>
            <real>0.61</real>
          </constant>
        </threshold>
        <threshold mcdaConcept="pref">
          <constant>
            <real>20.17</real>
          </constant>
        </threshold>
        <threshold mcdaConcept="veto">
          <constant>
            <real>98.61</real>
          </constant>
        </threshold>
      </thresholds>
    </criterion>
        
    <criterion id="g02" name="random cost criterion">
      <active>true</active>
      <scale>
        <quantitative>
        <preferenceDirection>minimum</preferenceDirection>
          <minimum>
            <real>0.00</real>
          </minimum>
          <maximum>
            <real>100.00</real>
          </maximum>
        </quantitative>
      </scale>
      <thresholds>
        <threshold mcdaConcept="ind">
          <constant>
            <real>19.81</real>
          </constant>
        </threshold>
        <threshold mcdaConcept="pref">
          <constant>
            <real>21.58</real>
          </constant>
        </threshold>
        <threshold mcdaConcept="veto">
          <constant>
            <real>98.33</real>
          </constant>
        </threshold>
      </thresholds>
    </criterion>
  </criteria>

  <criterionValue mcdaConcept="Majority threshold">
    <value><real>0.5</real></value>
  </criterionValue>
    
  <criteriaValues mcdaConcept="Importance" name="significance">
    <criterionValue>
      <criterionID>g01</criterionID>
      <value>
        <real>1.00</real>
      </value>
    </criterionValue>
    <criterionValue>
      <criterionID>g02</criterionID>
      <value>
        <real>1.00</real>
      </value>
    </criterionValue>
  </criteriaValues>

  <criteriaMatrix mcdaConcept="correlationTable">
    <description>
      <title>Ordinal Criteria Correlation Index</title>
      <comment>
        Generalisation of Kendall's tau to nested homogeneous semiorders.
      </comment>
    </description>
    <row>
      <criterionID>g01</criterionID>
      <column>
        <criterionID>g01</criterionID>
        <value>
          <real>1.00</real>
        </value>
      </column>
      <column>
      <criterionID>g02</criterionID>
        <value>
          <real>-0.33</real>
        </value>
      </column>
    </row>
    <row>
      <criterionID>g02</criterionID>
      <column>
        <criterionID>g01</criterionID>
        <value>
          <real>-0.33</real>
        </value>
      </column>
      <column>
        <criterionID>g02</criterionID>
        <value>
          <real>1.00</real>
        </value>
      </column>
    </row>
  </criteriaMatrix>
    
  <alternativesSets mcdaConcept="choices">
    <description>
      <title>Rubis Choice Recommendation</title>
      <comment>
        In decreasing order of determinateness. All values expressed in \%
      </comment>
    </description>
        
    <alternativesSet id="good_1" mcdaConcept="goodChoice">
      <description>
        <comment>Best choice</comment>
      </description>
      <element>
        <alternativeID>a03</alternativeID>
      </element>
      <value name="good choice scheme">
        <imageRef>http://mysite.com/myscheme.jpg</imageRef>
      </value>
      <values name="qualities">
        <value name="choiceSet independence">
          <real>100.00</real>
        </value>
        <value name="outranking">
          <real>74.00</real>
        </value>
        <value name="outranked">
          <real>0.00</real>
        </value>
        <value name="determinateness">
          <real>75.00</real>
        </value>
      </values>
    </alternativesSet>
  </alternativesSets>
      
  <alternativesComparisons mcdaConcept="outrankingDigraph" name="Stilde">
    <description>
      <title>Bipolar-valued Outranking Relation</title>
      <comment>Rubis Choice Recommendation Relation</comment>
    </description>
    <scale mcdaConcept="bipolar">
      <description>
        <subTitle>Valuation Domain</subTitle>
      </description>
      <quantitative>
        <preferenceDirection>maximum</preferenceDirection>
        <minimum>
            <real>-100</real>
        </minimum>
        <maximum>
            <real>100</real>
        </maximum>
      </quantitative>
    </scale>
    <pairs>
      <pair>
        <initial>
          <alternativeID>a01</alternativeID>
        </initial>
        <terminal>
          <alternativeID>a02</alternativeID>
        </terminal>
        <value>
          <real>100.00</real>
        </value>
      </pair>
      <pair>
        <initial>
          <alternativeID>a01</alternativeID>
        </initial>
        <terminal>
          <alternativeID>a03</alternativeID>
        </terminal>
        <value>
          <real>0.00</real>
        </value>
      </pair>
      <pair>
        <initial>
          <alternativeID>a02</alternativeID>
        </initial>
        <terminal>
          <alternativeID>a01</alternativeID>
        </terminal>
        <value>
          <real>0.00</real>
        </value>
      </pair>
      <pair>
        <initial>
          <alternativeID>a02</alternativeID>
        </initial>
        <terminal>
          <alternativeID>a03</alternativeID>
        </terminal>
        <value>
          <real>-100.00</real>
        </value>
      </pair>
      <pair>
        <initial>
          <alternativeID>a03</alternativeID>
        </initial>
        <terminal>
          <alternativeID>a01</alternativeID>
        </terminal>
        <value>
          <real>50.00</real>
        </value>
      </pair>
      <pair>
        <initial>
          <alternativeID>a03</alternativeID>
        </initial>
        <terminal>
          <alternativeID>a02</alternativeID>
        </terminal>
        <value>
          <real>100.00</real>
        </value>
      </pair>
    </pairs>
  </alternativesComparisons>
    
  <performanceTable id="rubis">
    <description>
      <title>Rubis Performance Table</title>
    </description>
    <alternativePerformances>
      <alternativeID>a01</alternativeID>
      <performance>
        <criterionID>g01</criterionID>
        <value>
          <real>72.10</real>
        </value>
      </performance>
      <performance>
        <criterionID>g02</criterionID>
        <value>
          <real>82.62</real>
        </value>
      </performance>
    </alternativePerformances>
    <alternativePerformances>
      <alternativeID>a02</alternativeID>
      <performance>
        <criterionID>g01</criterionID>
        <value>
          <real>4.74</real>
        </value>
      </performance>
      <performance>
        <criterionID>g02</criterionID>
        <value>
          <real>78.84</real>
        </value>
      </performance>
    </alternativePerformances>
    <alternativePerformances>
      <alternativeID>a03</alternativeID>
      <performance>
        <criterionID>g01</criterionID>
        <value>
          <real>62.88</real>
        </value>
      </performance>
      <performance>
        <criterionID>g02</criterionID>
        <value>
          <real>25.69</real>
        </value>
      </performance>
    </alternativePerformances>
  </performanceTable>
    
</xmcda:XMCDA>
\end{lstlisting}
}
\end{multicols}

%%%%%%%%%%%%%%%%%%%%%%%%%%%%%%%%%%%%%%%%%%%%%%%%%%%%%%%%%%%%%%%%%%%%%%%%%%%%


\newpage

\subsection{The output of the Kappalab web service}
\begin{multicols}{2}
{\code
\begin{lstlisting}[style=prototype, basicstyle=\tiny]
<?xml version="1.0" encoding="UTF-8"?>

<?xml-stylesheet href="xmcdaXSL.xsl" type="text/xsl" ?>
<?xml-stylesheet href="cssStyle.css" type="text/css" ?>

<xmcda:XMCDA xmlns:xmcda="http://www.decision-deck.org/2009/XMCDA-2.0.0" 
      xmlns:xsi="http://www.w3.org/2001/XMLSchema-instance" 
      xsi:schemaLocation="http://www.decision-deck.org/2009/XMCDA-2.0.0
      file:../XMCDA-2.0.0.xsd">

  <projectReference id="testProblem">
    <version>1.2</version>
    <creationDate>2009-03-20T22:24:02</creationDate>
  </projectReference>

  <methodParameters>
    <parameter name="identifiationMethod">
      <value><label>mini.var.capa.ident</label></value>
    </parameter>
    <parameter name="kadditivity">
      <value><integer>2</integer></value>
    </parameter>
  </methodParameters>
  
  <methodMessages>
    <logMessage name="executionStatus">
      <text>OK</text>
    </logMessage>
    <message name="objectiveFunction">
      <text>0.4899995</text>
    </message>
  </methodMessages>

  <alternatives mcdaConcept="Actions">
    <alternative id="x1">
      <active>true</active>
    </alternative>
    <alternative id="x2">
      <active>true</active>
    </alternative>
    <alternative id="x3">
      <active>true</active>
    </alternative>
    <alternative id="x4">
      <active>false</active>
    </alternative>
  </alternatives>

  <criteria>
    <description>
      <title>List of criteria</title>
    </description>

    <criterion id="g1">
      <description>
        <comment>Power in horsepowers</comment>
      </description>
      <scale>
        <quantitative>
          <preferenceDirection>maximum</preferenceDirection>
          <minimum>
            <real>0</real>
          </minimum>
          <maximum>
            <real>1</real>
          </maximum>
        </quantitative>
      </scale>
      <criterionFunction>
        <piecewiseLinear>
          <segment>
            <head>
              <abscissa>
                <real>50</real>
              </abscissa>
              <ordinate>
                <real>0</real>
              </ordinate>
            </head>
            <tail>
              <abscissa>
                <real>100</real>
              </abscissa>
              <ordinate>
                <real>0.125</real>
              </ordinate>
            </tail>
          </segment>
          <segment>
            <head>
              <abscissa>
                <real>100</real>
              </abscissa>
              <ordinate>
                <real>0.125</real>
              </ordinate>
            </head>
            <tail>
              <abscissa>
                <real>200</real>
              </abscissa>
              <ordinate>
                <real>1</real>
              </ordinate>
            </tail>
          </segment>
        </piecewiseLinear>
      </criterionFunction>
    </criterion>

    <criterion id="g2">
      <description>
        <comment>Design appreciation</comment>
      </description>
      <scale>
        <quantitative>
          <preferenceDirection>maximum</preferenceDirection>
          <minimum>
            <real>0</real>
          </minimum>
          <maximum>
            <real>1</real>
          </maximum>
        </quantitative>
      </scale>
      <criterionFunction>
        <points>
          <point>
            <abscissa>
              <rankedLabel>
                <label>bad</label>
                <rank>3</rank>
              </rankedLabel>
            </abscissa>
            <ordinate>
              <real>0</real>
            </ordinate>
          </point>
          <point>
            <abscissa>
              <rankedLabel>
                <label>medium</label>
                <rank>2</rank>
              </rankedLabel>
            </abscissa>
            <ordinate>
              <real>0.3</real>
            </ordinate>
          </point>
          <point>
            <abscissa>
              <rankedLabel>
                <label>good</label>
                <rank>1</rank>
              </rankedLabel>
            </abscissa>
            <ordinate>
              <real>1</real>
            </ordinate>
          </point>
        </points>
      </criterionFunction>
    </criterion>

    <criterion id="g3">
      <description>
        <comment>Price in Euros</comment>
      </description>
      <scale>
        <quantitative>
          <preferenceDirection>maximum</preferenceDirection>
          <minimum>
            <real>0</real>
          </minimum>
          <maximum>
            <real>1</real>
          </maximum>
        </quantitative>
      </scale>
      <criterionFunction>
        <linear>
          <slope>
            <rational>
              <numerator>-1</numerator>
              <denominator>15000</denominator>
            </rational>
          </slope>
          <intercept>
            <real>2</real>
          </intercept>
        </linear>
      </criterionFunction>
    </criterion>
  </criteria> 

  <criteriaValues mcdaConcept="capacityMoebius">
    <criterionValue>
      <criteriaSet>
        <element>
          <criterionID>g1</criterionID>
        </element>
      </criteriaSet>
      <value>
        <real>0.498899898572209</real>
      </value>
    </criterionValue>

    <criterionValue>
      <criteriaSet>
        <element>
          <criterionID>g2</criterionID>
        </element>
      </criteriaSet>
      <value>
        <real>0.0236248732152609</real>
      </value>
    </criterionValue>

    <criterionValue>
      <criteriaSet>
        <element>
          <criterionID>g3</criterionID>
        </element>
      </criteriaSet>
      <value>
        <real>137365.556042755715066</real>
      </value>
    </criterionValue>
  </criteriaValues>

  <criteriaComparisons mcdaConcept="importancePreorderConstraints">
    <comparisonType>geq</comparisonType>
    <pairs>
      <pair>
        <initial>
          <criterionID>g1</criterionID>
        </initial>
        <terminal>
          <criterionID>g2</criterionID>
        </terminal>
        <value>
          <real>0.1</real>
        </value>
      </pair>
      <pair>
        <initial>
          <criterionID>g3</criterionID>
        </initial>
        <terminal>
          <criterionID>g2</criterionID>
        </terminal>
        <value>
          <real>0.1</real>
        </value>
      </pair>
    </pairs>
  </criteriaComparisons>

  <alternativesValues mcdaConcept="overallValues">
    <alternativeValue>
      <alternativeID>x1</alternativeID>
      <value>
        <real>0.598899898572209</real>
      </value>
    </alternativeValue>
    <alternativeValue>
      <alternativeID>x2</alternativeID>
      <value>
        <integer>24</integer>
      </value>
    </alternativeValue>
    <alternativeValue>
      <alternativeID>x3</alternativeID>
      <value>
        <real>0.398899898572209</real>
      </value>
    </alternativeValue>
  </alternativesValues>

  <alternativesComparisons mcdaConcept="preorderConstraints">
    <description>
      <subTitle>
        Preorder constraints on the overall values of the alternatives
      </subTitle>
    </description>
    <comparisonType>preference</comparisonType>
    <pairs>
      <pair>
        <initial>
          <alternativeID>x1</alternativeID>
        </initial>
        <terminal>
          <alternativeID>x2</alternativeID>
        </terminal>
        <value>
          <real>0.1</real>
        </value>
      </pair>
      <pair>
        <initial>
          <alternativeID>x2</alternativeID>
        </initial>
        <terminal>
          <alternativeID>x3</alternativeID>
        </terminal>
        <value>
          <real>0.1</real>
        </value>
      </pair>
    </pairs>
  </alternativesComparisons>

  <performanceTable id="original">
    <description>
      <title>Original performance table</title>
    </description>
    <alternativePerformances>
      <alternativeID>x2</alternativeID>
      <performance>
        <criterionID>gd2</criterionID>
        <value>
          <rankedLabel>
            <label>bad</label>
            <rank>3</rank>
          </rankedLabel>
        </value>
      </performance>
      <performance>
        <criterionID>g3</criterionID>
        <value>
          <real>30000</real>
        </value>
      </performance>
      <performance>
        <criterionID>g1</criterionID>
        <value>
          <real>200</real>
        </value>
      </performance>
    </alternativePerformances>
    <alternativePerformances>
      <alternativeID>x1</alternativeID>
      <performance>
        <criterionID>g3</criterionID>
        <value>
          <real>15000</real>
        </value>
      </performance>
      <performance>
        <criterionID>g1</criterionID>
        <value>
          <real>50</real>
        </value>
      </performance>
      <performance>
        <criterionID>g2</criterionID>
        <value>
          <rankedLabel>
            <label>medium</label>
            <rank>2</rank>
          </rankedLabel>
        </value>
      </performance>
    </alternativePerformances>
    <alternativePerformances>
      <alternativeID>x3</alternativeID>
      <performance>
        <criterionID>g3</criterionID>
        <value>
          <real>22500</real>
        </value>
      </performance>
      <performance>
        <criterionID>g1</criterionID>
        <value>
          <real>100</real>
        </value>
      </performance>
      <performance>
        <criterionID>g2</criterionID>
        <value>
          <rankedLabel>
            <label>good</label>
            <rank>1</rank>
          </rankedLabel>
        </value>
      </performance>
    </alternativePerformances>
  </performanceTable>

</xmcda:XMCDA>
\end{lstlisting}
}
\end{multicols}


\end{document}













